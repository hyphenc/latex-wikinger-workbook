%!TEX TS-program = pdflatex
% wikinger workbook
% https://tex.stackexchange.com/questions/1319/showcase-of-beautiful-typography-done-in-tex-friends/74609#74609
\documentclass[12pt,a4paper,ngerman]{book}
\usepackage[demo]{graphicx}
\usepackage[hang,splitrule]{footmisc}
\usepackage[ngerman]{babel}
\usepackage[T1]{fontenc}
\usepackage[utf8]{inputenc}
\usepackage{amssymb}
\usepackage{booktabs}
\usepackage{chngcntr}
\usepackage{color}
\usepackage{afterpage}
\usepackage{fancyhdr}
\usepackage{fourier-orns}
\usepackage{geometry}
\usepackage{graphicx}
\usepackage{hyphenat}
\usepackage{lettrine}
\usepackage{lipsum}
\usepackage{setspace}
\usepackage{tabularx}
\usepackage{titlesec}
\usepackage{wrapfig}
\usepackage{etoolbox}
\usepackage[final,
stretch=10,protrusion=true,
tracking=true,expansion=true]{microtype}

\renewcommand{\familydefault}{pplj} 
\renewcommand{\footnoterule}{\vspace{0.5em}\noindent{\line(1,0){120}} \vspace{.5em} }
\renewcommand{\LettrineFontHook}{\initfamily}
\newcommand{\flettrine}[2]{\lettrine[lines=2, depth=0, loversize=0.25, nindent=0.69pt, lraise=0.15]{\initfamily{#1}}{#2}}
\newcommand{\ornpar}{\noindent {\hrulefill \raisebox{-1.9pt}[10pt][10pt]{\textxswup}}}
\newcommand{\ornimpar}{\raisebox{-1.9pt}[10pt][10pt]{\textxswup} \hrulefill}
\newcommand{\estcab}[1]{\itshape{\nouppercase #1}}
\newcommand*\initfamily{\usefont{U}{GotIn}{xl}{n}}
\newcommand\blankpage{\null \thispagestyle{empty} \addtocounter{page}{-1} \newpage}

\geometry{
tmargin=3cm,bmargin=3cm,
lmargin=3cm,rmargin=2cm,
headheight=1.5cm,headsep=0.8cm,
footskip=0.5cm}

\setlength{\parskip}{1.3ex plus 0.2ex minus 0.2ex}
% ^^^ \usepackage[parfill]{parskip} ? TODO
\setlength{\parindent}{1em}
\setlength{\footnotemargin}{0.3cm}
\setlength{\footnotesep}{0.4cm} 
\addtolength{\footskip}{0.5cm}

\newcommand{\fchapter}[1]{\chapter{#1}\thispagestyle{chapterstyle}}

% https://www.overleaf.com/learn/latex/headers_and_footers#Reference_guide
\fancyhf{}

\fancypagestyle{chapterstyle}{ % für kapitelbeginn
  \renewcommand{\headrulewidth}{0pt}
  \fancyhead{}
  \fancyfoot[LE,RO]{\thepage}
}

\fancypagestyle{normalpage}{ % normale seiten
  \fancyhead[LE,RO]{\textit{\nouppercase{\leftmark}}}
  \fancyhead[RE,LO]{\textit{\nouppercase{\rightmark}}}
  \fancyfoot[LE,RO]{\thepage}
}

\pagestyle{normalpage}
\renewcommand{\chaptermark}[1]{\markboth{#1}{}}
\renewcommand{\sectionmark}[1]{\markright{#1}{}}

\counterwithout{figure}{chapter} % continuous figure/table numbering
\counterwithout{table}{chapter} % continuous figure/table numbering

\input GotIn.fd % lettrine font
\setstretch{1} % 1x line spacing for toc
\hyphenation{Mathe-matik wieder-gewinnen} % richtige worttrennung

% space between 'chap #' and chap-title (https://latex.org/forum/viewtopic.php?t=21666)
\titleformat{\chapter}[display]{\normalfont\huge\bfseries}{\chaptertitlename\ \thechapter}{0pt}{\Huge}

%%%%%%%%%%%%%%%%%%%%%%%%%%%%%%%%%%%%%%%%%%%%%
%%%%%%%%%%%%%%%%%%%%%%%%%%%%%%%%%%%%%%%%%%%%%

% TODO aufgaben environment. ähnlich zu fußnoten
% TODO fußnoten in den jew. rand
% https://zapier.com/blog/download-images-google-doc-word/
% \glqq \grqq richtige quotes
% titelseite in gimp
% lösungsheft!
% https://tex.stackexchange.com/questions/23860/how-to-include-a-picture-over-two-pages-left-part-on-left-side-right-on-right ?
% doppelseitig: mit short edge drucken

\begin{document}

\author{titelseite}
\date{titelseite}
\title{titelseite}

\afterpage{\blankpage}
% NOTE für leseversion aus
\maketitle
\thispagestyle{empty}
\tableofcontents
% TODO heading weiter oben
\thispagestyle{empty}
\newpage
\thispagestyle{empty}

\setstretch{1.125} % 1.25 line spacing for text

\fchapter{Einleitung}

\section{Wer waren die Wikinger?}

\flettrine{D}{er Wikinger:} Einen eisernen Helm auf dem Kopf, ein schweres Schwert in den Händen, plumpe Kleider am Körper und direkt auf dem Weg zum nächsten Raubzug.
Klingt schräg, aber ist doch das Bild, das jedem bei dem Gedanken an die alten Wikinger in den Kopf kommt. Die Frage ist jedoch: Stimmt dieses vermeintliche Wissen über die Wikinger?
Waren sie wirklich so barbarisch? Waren die Wikinger kopflose Kämpfer, denen es nur nach Reichtum schmachtete?\\\\
Die Antwort ist gar nicht so einfach. Man könnte sagen ja und nein. Beginnen wir ganz am Anfang. Das Wikingerzeitalter begann im Jahr 793 und erstreckte sich bis zum Jahr 1066.
Verschiedenste Männer kamen verstreut aus dem skandinavischen Raum, also Norwegen, Schweden und Dänemark, es war also ein Zusammentreffen unterschiedlichster Kulturen und Individuen.
Und diese waren keinesfalls ausnahmslos Krieger oder Kämpfer, in dem bunten Gemisch aus Menschen fanden sich sowohl zahlreiche Händler, als auch Bauern, genauso wie Seemänner und unter anderem auch nur einfache Siedler.
Sie begannen die weiten Gewässer zu erkunden, neues Gebiet zu entdecken und auch um ihr Überleben zu kämpfen. Es entwickelten sich gesellschaftliche Stämme und Dörfer.
Ein Beispiel dafür kann man noch heute fast eintausend Jahre später in der Nähe von Schleswig begutachten: das 770 gegründete Haithabu. Zu damaliger Zeit war dieser zentrale,
zwischen der Nord- und Ostsee gelegene Ort ein wichtiger Handelsort und Hauptumschlagplatz für die Wikinger aus Skandinavien, dem Baltikum und Westeuropa. Mit verschiedenen Rohstoffen wie beispielsweise Eisen,
das zu jener Zeit sogar recht selten und kostbar war, wurden Waffen hergestellt, mit denen Kämpfe ausgetragen wurden.\\
Auf ihren Reisen auf dem Meer entdeckten sie nicht nur neues Gebiet, sondern trafen auch auf andere Gesellschaften. Bei diesen Begegnungen gingen sie rücksichtslos vor, raubten andere Völker aus und zerstörten anderes Eigentum.
Solche Kämpfe waren üblich. Außerdem versklavten die Wikinger die Überlebenden und nahmen sie mit auf ihre Raubzüge. Andere Sklaven wurden von diesen in das Heimatdorf der Wikinger gebracht, damit jene dort bei den zu erledigenden Aufgaben halfen.
Zum anderen waren die Wikinger auch Händler, die mit verschiedensten Waren auf Reisen aber auch im Heimatdorf handelten.\\
Auf der anderen Seite sollte durchaus erwähnt werden, dass die Gesellschaft der Wikinger für die damalige Zeit recht fortgeschritten war. Zum Beispiel gab es die Möglichkeit, seine Meinung in der Öffentlichkeit relativ frei zu äußern.
Eigentlich kann man aber nicht von einer Gesellschaft sprechen, sondern eher von mehreren sogenannten Sippen und Familienverbänden. In einer Sippe lebten teilweise bis zu drei Generationen gemeinsam: Die Großeltern, die Eltern und die Kinder.
Zusammen hausten sie in einem großen Raum, welchen wir heute als das typische Haus der Wikinger bezeichnen. Gerade Kinder und Eltern waren einander im Alltag wichtige Partner. Während Mutter und Vater ihre Aufgaben erledigten,
lernten Tochter und Sohn von ihnen und halfen ihren Eltern. Somit stand die Zukunft für das jeweilige Geschlecht bereits früh fest; das Mädchen lernte die Arbeiten der Mutter kennen, zum Beispiel die Fertigung von Kleidung, und mit der Zeit auch, sie zu übernehmen.
Genauso wie die Jungen auf der anderen Seite sich die Tätigkeiten der Väter aneigneten, zum Beispiel der Umgang mit der Waffe auf Raubzügen, auf die sie ihre Väter manchmal begleiteten.\\
Auf politischer Ebene musste man den Wikingern sicherlich auch eine gewisse Anerkennung entgegenbringen, denn Gesetze und auch Rechte waren sehr wichtige Werte für die Wikinger. Zum Beispiel tagten die Wikinger regelmäßig zusammen in Form einer Volksversammlung,
welche das “Thing” genannt wurde. Die Termine waren festgelegt und die Versammlungen, in denen man über politische Angelegenheiten, Recht und Gesetz abstimmte und beriet, fanden immer unter freiem Himmel statt.\\
Außerdem war eine Hochzeit zwischen Mann und Frau offiziell durchaus möglich, genauso aber auch, sich wieder voneinander scheiden zu lassen. 
Für die Zivilisiertheit der Wikinger sprach außerdem ihre Intention, sich nach außen hin gut und ordentlich zu präsentieren, denn die Wikinger legten Wert auf ihr Äußeres. Dazu zählte die Art sich zu kleiden,
ein gewisses Maß an Hygiene und auch Schmuck war bereits damals ein wichtiges Accessories. Zahlreiche Goldschmiedearbeiten und Runensteine, die man später in vielen der Gräbern fand, zeugten von der geistigen Feinheit der Wikinger. 



\section{Zeitlicher Überblick}

% https://tex.stackexchange.com/questions/462984/fancy-vertical-timeline
% https://tex.stackexchange.com/questions/544122/vertical-timeline-that-uses-width-of-the-page


\fchapter{Der typische Wikinger}

\section{Kleidung}
\section{Alltagskleidung}

\section{Kriegerkleidung}

\section{Haus und Hof} % TODO für "Wohnsituation, Nahrung, etc" ??

\section{Alltag}

\lipsum[1-2]

\section{Waffen}

\lipsum[1-2]

\section{Alltag} % keine unterteilung in unterkapitel

\lipsum[6]

\fchapter{Wikinger um die Welt}

\section{Karte} %TODO: name!

\section{Die Wikinger in England}

\lipsum[3]

\fchapter{Die Wikingergesellschaft}

\section{Ständeordnung}

\lipsum[6]

\section{Gemeinschaft}

\lipsum[6]

\section{Die Rolle der Frau}
%TODO mit oder ohne "die"?

\lipsum[6]

\fchapter{Zusammenfassung}
%TODO: fix name

\section{Wiederholung}
%TODO: name, hört sich zu expansiv an

\lipsum[6]

\section{Ende}

\lipsum[6]

\newpage % für quellen
\section{Quellenangaben}

\end{document}