%!TEX TS-program = pdflatex
% wikinger workbook
\documentclass[12pt,a4paper,ngerman,openany]{book}
\usepackage{graphicx}
\usepackage[hang,splitrule]{footmisc}
\usepackage[ngerman]{babel}
\usepackage[parfill]{parskip}
\usepackage[T1]{fontenc}
\usepackage[utf8]{inputenc}
\usepackage{afterpage}
\usepackage{amssymb}
\usepackage{booktabs}
\usepackage{chngcntr}
\usepackage{color}
\usepackage{etoolbox}
\usepackage{fancyhdr}
\usepackage{geometry}
\usepackage{graphicx}
\usepackage{hyphenat}
\usepackage{lettrine}
\usepackage{paralist}
\usepackage{pdfpages}
\usepackage{setspace}
\usepackage{tabularx}
\usepackage{tcolorbox}
\usepackage{titlesec}
\usepackage{wrapfig}
\usepackage[final,
stretch=10,protrusion=true,
tracking=true,expansion=true]{microtype}

\renewcommand{\familydefault}{pplj} 
\renewcommand{\footnoterule}{\vspace{0.5em}\noindent{\line(1,0){120}} \vspace{.5em} }
\renewcommand{\LettrineFontHook}{\initfamily}
\newcommand{\flettrine}[2]{\lettrine[lines=2, depth=0, loversize=0.25, nindent=0.69pt, lraise=0.15]{\initfamily{#1}}{#2}}
\newcommand{\estcab}[1]{\itshape{\nouppercase #1}}
\newcommand*\initfamily{\usefont{U}{GotIn}{xl}{n}}
\newcommand\blankpage{\null \thispagestyle{empty} \addtocounter{page}{-1} \newpage}
\newcommand{\aufgaben}[1]{
  \begin{tcolorbox}
    \textbf{Aufgaben}
    \begin{enumerate}
      #1
    \end{enumerate}
  \end{tcolorbox}
}
\newcommand{\erklaer}[2]{\leavevmode\marginpar{\footnotesize \textbf{#1:}\\#2}\ignorespaces}

\geometry{
tmargin=0cm,bmargin=3cm,
lmargin=3cm,rmargin=2cm,
headheight=1.5cm,headsep=0.8cm,
footskip=1cm}

\setlength{\parskip}{1.3ex plus 0.2ex minus 0.2ex}
\setlength{\parindent}{0cm}
\setlength{\footnotemargin}{0.3cm}
\setlength{\footnotesep}{0.4cm} 
\addtolength{\footskip}{0.5cm}
\setdefaultleftmargin{0.667cm}{}{}{}{}{}

\newcommand{\fchapter}[1]{\chapter{#1}\thispagestyle{chapterstyle}}

\fancyhf{}

\fancypagestyle{chapterstyle}{ % für kapitelbeginn
  \renewcommand{\headrulewidth}{0pt}
  \fancyhead{}
  \fancyfoot[LE,RO]{\thepage}
}

\fancypagestyle{normalpage}{ % normale seiten
  \fancyhead[LE,RO]{\textit{\nouppercase{\leftmark}}}
  \fancyhead[RE,LO]{\textit{\nouppercase{\rightmark}}}
  \fancyfoot[LE,RO]{\thepage}
}

\pagestyle{normalpage}
\renewcommand{\chaptermark}[1]{\markboth{#1}{}}
\renewcommand{\sectionmark}[1]{\markright{#1}{}}

\counterwithout{figure}{chapter} % continuous figure/table numbering
\counterwithout{table}{chapter} % continuous figure/table numbering

\input GotIn.fd % lettrine font
\setstretch{1.125} % 1.125x line spacing
\hyphenation{Mathe-matik wieder-gewinnen} % silbentrennung
\graphicspath{{./img/}}

% space between 'chap #' and chap-title (https://latex.org/forum/viewtopic.php?t=21666)
\titleformat{\chapter}[display]{\normalfont\huge\bfseries}{\chaptertitlename\ \thechapter}{0pt}{\Huge}

%%%%%%%%%%%%%%%%%%%%%%%%%%%%%%%%%%%%%%%%%%%%%
%%%%%%%%%%%%%%%%%%%%%%%%%%%%%%%%%%%%%%%%%%%%%

% %NOTE https://zapier.com/blog/download-images-google-doc-word/
% %TODO \glqq \grqq richtige quotes
% https://tex.stackexchange.com/questions/23860/how-to-include-a-picture-over-two-pages-left-part-on-left-side-right-on-right ?
% doppelseitig: mit short edge drucken
% TODO hunspell -d de_DE -t --check-apostrophe --check-url main.tex
% pdfbook2 -t 0 -b 0 -i 0 -o 0 main.pdf
% TODO - und – durch \textemdash ersetzen

% TODO tabelle mit waffen
% https://tex.stackexchange.com/questions/86006/adding-image-next-to-text-in-table

% TODO danegeld, bondi, boendr, etc -> \textit

% TODO waffen, kleidungs tabellen sonst auf die empty pages hinten

\begin{document}

\includepdf[pages=-]{wikinger-workbook-front.pdf}
\thispagestyle{empty}
\newpage

\afterpage{\blankpage}
\tableofcontents
\thispagestyle{empty}

\newgeometry{top=2cm, bottom=2.5cm,% for text
outer=3.75cm, inner=2cm,%
heightrounded, marginparwidth=2.8cm,%
marginparsep=0.3cm}

\fchapter{Einleitung}

\section{Wer waren die Wikinger?}

\flettrine{D}{er Wikinger:} Einen eisernen Helm auf dem Kopf, ein schweres Schwert in den Händen, plumpe Kleider am Körper und direkt auf dem Weg zum nächsten Raubzug.
Klingt schräg, aber ist doch das Bild, das jedem bei dem Gedanken an die alten Wikinger in den Kopf kommt. Die Frage ist jedoch: Stimmt dieses vermeintliche Wissen über die Wikinger?
Waren sie wirklich so barbarisch? Waren die Wikinger kopflose Kämpfer, denen es nur nach Reichtum schmachtete?\\\\
Die Antwort ist gar nicht so einfach. Man könnte sagen ja und nein. Beginnen wir ganz am Anfang. Das Wikingerzeitalter begann im Jahr 793 und erstreckte sich bis zum Jahr 1066.
Verschiedenste Männer kamen verstreut aus dem skandinavischen Raum, also Norwegen, Schweden und Dänemark, es war also ein Zusammentreffen unterschiedlichster Kulturen und Individuen.
Und diese waren keinesfalls ausnahmslos Krieger oder Kämpfer, in dem bunten Gemisch aus Menschen fanden sich sowohl zahlreiche Händler, als auch Bauern, genauso wie Seemänner und unter anderem auch nur einfache Siedler.
Sie begannen die weiten Gewässer zu erkunden, neues Gebiet zu entdecken und auch um ihr Überleben zu kämpfen. Es entwickelten sich gesellschaftliche Stämme und Dörfer.
Ein Beispiel dafür kann man noch heute fast eintausend Jahre später in der Nähe von Schleswig begutachten: das 770 gegründete Haithabu. Zu damaliger Zeit war dieser zentrale,
zwischen der Nord- und Ostsee gelegene Ort ein wichtiger Handelsort und Hauptumschlagplatz für die Wikinger aus Skandinavien, dem Baltikum und Westeuropa. Mit verschiedenen Rohstoffen wie beispielsweise Eisen,
das zu jener Zeit sogar recht selten und kostbar war, wurden Waffen hergestellt, mit denen Kämpfe ausgetragen wurden.\\ 
Auf ihren Reisen auf dem Meer entdeckten sie nicht nur neues Gebiet, sondern trafen auch auf andere Gesellschaften. Bei diesen Begegnungen gingen sie rücksichtslos vor, raubten andere Völker aus und zerstörten anderes Eigentum.
Solche Kämpfe waren üblich. Außerdem versklavten die Wikinger die Überlebenden und nahmen sie mit auf ihre Raubzüge. Andere Sklaven wurden von diesen in das Heimatdorf der Wikinger gebracht, damit jene dort bei den zu erledigenden Aufgaben halfen.
Zum anderen waren die Wikinger auch Händler, die mit verschiedensten Waren auf Reisen aber auch im Heimatdorf handelten.\\
Auf der anderen Seite sollte durchaus erwähnt werden, dass die Gesellschaft der Wikinger für die damalige Zeit recht fortgeschritten war. Zum Beispiel gab es die Möglichkeit, seine Meinung in der Öffentlichkeit relativ frei zu äußern.
Eigentlich kann man aber nicht von einer Gesellschaft sprechen, sondern eher von mehreren sogenannten Sippen und Familienverbänden. In einer Sippe lebten teilweise bis zu drei Generationen gemeinsam: Die Großeltern, die Eltern und die Kinder.
Zusammen hausten sie in einem großen Raum, welchen wir heute als das typische Haus der Wikinger bezeichnen. Gerade Kinder und Eltern waren einander im Alltag wichtige Partner. Während Mutter und Vater ihre Aufgaben erledigten,
lernten Tochter und Sohn von ihnen und halfen ihren Eltern. Somit stand die Zukunft für das jeweilige Geschlecht bereits früh fest; das Mädchen lernte die Arbeiten der Mutter kennen, zum Beispiel die Fertigung von Kleidung, und mit der Zeit auch, sie zu übernehmen.
Genauso wie die Jungen auf der anderen Seite sich die Tätigkeiten der Väter aneigneten, zum Beispiel der Umgang mit der Waffe auf Raubzügen, auf die sie ihre Väter manchmal begleiteten.\\
Auf politischer Ebene musste man den Wikingern sicherlich auch eine gewisse Anerkennung entgegenbringen, denn Gesetze und auch Rechte waren sehr wichtige Werte für die Wikinger. Zum Beispiel tagten die Wikinger regelmäßig zusammen in Form einer Volksversammlung,
welche das “Thing” genannt wurde. Die Termine waren festgelegt und die Versammlungen, in denen man über politische Angelegenheiten, Recht und Gesetz abstimmte und beriet, fanden immer unter freiem Himmel statt.\\
Außerdem war eine Hochzeit zwischen Mann und Frau offiziell durchaus möglich, genauso aber auch, sich wieder voneinander scheiden zu lassen. 
Für die Zivilisiertheit der Wikinger sprach außerdem ihre Intention, sich nach außen hin gut und ordentlich zu präsentieren, denn die Wikinger legten Wert auf ihr Äußeres. Dazu zählte die Art sich zu kleiden,
ein gewisses Maß an Hygiene und auch Schmuck war bereits damals ein wichtiges Accessoire. Zahlreiche Goldschmiedearbeiten und Runensteine, die man später in vielen der Gräbern fand, zeugten von der geistigen Feinheit der Wikinger. 

\aufgaben {
  \item Lies den Text und notiere dir Stichworte, um dir einen Überblick zu schaffen.
}

\pagebreak

\section{Zeitlicher Überblick}

% https://tex.stackexchange.com/questions/462984/fancy-vertical-timeline
% https://tex.stackexchange.com/questions/544122/vertical-timeline-that-uses-width-of-the-page
% zur not selber boxen und linien draw-en

\begin{tcolorbox}[sharp corners, title=08. Juni 793]
Erster historisch festgehaltene Angriff auf die Abtei Lindisfarne in Northumberland, Nordengland. Hier wurden Reliquien des Heiligen Cuthbert aufbewahrt. Die angreifenden Wikinger plünderten das Kloster und legten es in Schutt und Asche.\\
Bereits vorher gab es schon Aktivitäten, jedoch wurden diese nicht historisch festgehalten.
\end{tcolorbox}

\begin{tcolorbox}[sharp corners, title=Ungefähr 800 bis 850]
In dieser Anfangszeit gab es meist unkoordinierte Angriffe auf Orte ohne gute Verteidigung und mit viel Reichtum, wo es also viel potenzielle Beute gab. Oft wurden Klöster, Abteien und offene Städte (Städte ohne Verteidigung) überfallen.
\end{tcolorbox}

%TODO schutzgeld erklären, inline

\begin{tcolorbox}[sharp corners, title=Ungefähr 850 bis 900]
In dieser sehr aktiven Phase wurden die Wikinger sich langsam ihres kriegerischen Potenzials bewusst. Die Angriffe wurden jetzt besser geplant, laufen koordinierter ab und die allgemeine Bevölkerung anderer Länder hatte große Angst vor den Plünderern.
Jedoch wurden stärkere Nationen, die sich gegen den Wikingeransturm verteidigt haben, wie zum Beispiel Südengland, zunehmend in Ruhe gelassen. Daraus kann man erkennen, dass die Wikinger keine richtigen Kriegsherren waren und auch öfters größere Schlachten verloren haben. Zumal das Wikingervolk auch zu klein war, um Truppen aufzustellen, die groß genug wären, um sich mit den stärkeren Gegnern zu messen.
In dieser Zeit forderten die Wikinger auch das sogenannte “danegeld” (“Bezahlung an die Dänen”). Dieses immer steigende Schutzgeld wurde gezahlt, damit die Wikinger Dänemark nicht überfallen.
In dieser Zeit kristallisierten sich auch die wichtigsten Handelsrouten heraus, wie zum Beispiel die West- und Nordroute.
\end{tcolorbox}

\begin{tcolorbox}[sharp corners, title=Ungefähr 900 bis 980]
Zunehmend entstehen feste Niederlassungen und Territorien, vor allem in Skandinavien, aber auch beispielsweise in der Normandie (Nordwestfrankreich), Grönland, in Teilen Englands und Novgorod (Westrussland), zu welchen auch Handelsrouten bestanden. Falls die Wikinger sich aber, wie oben teilweise genannt, in anderen Ländern niederließen, mussten sie auch die dortigen Gesetze befolgen und sich dort eingliedern, wozu auch der Verteidigungsdienst für das “neue” Vaterland zählte. Dadurch verschwand in diesen Fällen “der Wikinger” nach ein paar Generationen.
\end{tcolorbox}

\begin{tcolorbox}[sharp corners, title=Ungefähr 980 bis 1066]
Hier ging es meist nur noch um die restlichen Wikinger im Nordwesten Dänemarks und Südosten Schwedens. In Dänemark versuchten sie, für eine kurze Zeit erfolgreich, die Herrschaft an sich zu reißen, während in Schweden lange Expeditionen nach Zentralasien gemacht wurden. Jedoch verloren die Wikinger zunehmend an Bedeutung, nicht nur weil das Wikingervolk immer kleiner wurde. Als größter Faktor für das “Verschwinden” des Wikingers ist die Christianisierung Nordeuropas zu nennen, durch welche auch zentrale Herrscher (Könige) großer Gebiete begünstigt wurden, zumal die einzelnen Wikingergruppen nie solche großen Gebiete kontrolliert haben, sondern eher lokale, kleinere Territorien beanspruchten.
\end{tcolorbox}

\vspace{0.66cm}

\aufgaben{
  \item Ordne die folgenden Szenarien den verschiedenen Zeitperioden zu.
}

\begin{itemize}
\item Björn und Helga bauen sich mit der Hilfe ihrer Familien ein Haus.
\item Sven und einige Freunde aus dem Dorf lagern etwas Proviant für den morgigen Überfall auf ein kleines Kloster ein.
\item Sigmund und Thorleif arbeiten einen detaillierten Plan für einen Raubzug aus, an dem mehrere Dörfer beteiligt sind.
\item Der Bischof eines englisches Stifts bekommt zunehmend Angst, dass seine Abtei überfallen werden könnte.
\item Ein paar Wikinger bereiten sich auf eine lange Expedition nach Asien vor.
\item Der Jarl Olaf gibt seine Territorien auf und schließt sich dem neuen Königreich an.
\item Einige Königshäuser beginnen, den Wikingern Schutzgeld zu bezahlen.
\end{itemize}

\fchapter{Der typische Wikinger}

\section{Alltagskleidung}
TEXT

\section{Kriegerkleidung}
TEXT

\aufgaben{
  \item Schneide die Erklärtexte aus und ordne sie den richtigen Kriegerklamotten zu.
}

\section{Waffen}
TEXT

\aufgaben{
  \item Schneide die Bilder aus und ordne sie dem passenden Erklärtext zu.
}

\section{Leben im Dorf}

\aufgaben{
  \item Bevor du weitermachst: Stelle dir vor, du lebtest in einem Wikingerdorf, wie könnte dein Alltag aussehen? Wie würde dein Zuhause aussehen? Beschreibe deine Ideen in einem Text.
  \item Lies die beiden Texte.
}

\subsection{Haus und Hof}
TEXT

\subsection{Alltagsablauf}
TEXT

\aufgaben{
  \item Nutze die Informationen aus dem Text und schildere deinen Tagesablauf in einem Tagebucheintrag.
  \item Vergleiche die Informationen aus den Texten mit deinen Erwartungen und nenne Beispiele, wie sich der Alltag zu heute verändert hat.
}

\fchapter{Wikinger um die Welt}

\section{Geografischer Überblick}

Williams Karte

\aufgaben{
  \item Bearbeite die Lückentexte mithilfe der Informationen aus der Karte.
}

\section{Die Wikinger in England}
TEXT

\aufgaben{
  \item Lies den Text und erstelle eine Chronologie der Ereignisse.
  \item Begründe warum England für die Wikinger als Angriffsziel so attraktiv war.
}

\fchapter{Die Wikingergesellschaft}

\section{Ständeordnung}
TEXT

\aufgaben{
  \item Begründe, ob du dich in der Wikingergesellschaft wohlfühlen würdest und beziehe Sichtweisen aus den verschiedenen Ständen ein.
  \item Bewerte die Fairness und soziale Mobilität (Beweglichkeit in Bezug auf die soziale Stellung) in der Gesellschaft.
}

\section{Gemeinschaft und Gesellschaftsbild}
TEXT

\aufgaben{
  %TODO fragen zum text
  \item Frage (was ist eine blutsbrüderschaft?, X, Y) + Freiform Antwortmgl.
  \item Beurteile ob die “Gemeinschaft” heute noch so einen hohen Stellenwert wie bei den Wikingern hat und inwiefern sich der Begriff gewandelt hat (z.B. durch Social Media).
}

\section{Die Rolle der Frau}

\aufgaben{
  \item Fasse die Aufgaben der Frau zusammen.
  \item Bewerte mithilfe deiner Notizen zu voriger Aufgabe die Rolle der Frau (z.B. in Bezug auf Gleichberechtigung).
  \item Arbeite Gemeinsamkeiten und Unterschiede in Gegenüberstellung zu der Frau heute heraus.
}

\fchapter{Schluss}

\section{Zusammenfassung}

\newpage % für quellen
\section{Quellenangaben}

\textbf{Kapitel 1}
\begin{itemize}
  \item 
\end{itemize}

\textbf{Kapitel 2}
\begin{itemize}
  \item 
\end{itemize}

\textbf{Kapitel 3}
\begin{itemize}
  \item 
\end{itemize}

\textbf{Kapitel 4}
\begin{itemize}
  \item 
\end{itemize}

\textbf{Kapitel 5}
\begin{itemize}
  \item 
\end{itemize}


% TODO so viele leere seiten einfügen wie nötig
% TODO how to avoid blank pages --signature ?
\afterpage{\blankpage}
\afterpage{\blankpage}
\afterpage{\blankpage}
\afterpage{\blankpage}
\afterpage{\blankpage}

\newpage
\vspace*{24cm} Erstellt von Alexa, Mareike, Nils und William // Projektkurs Wikinger 2020
\thispagestyle{empty}
\newpage
\includepdf[pages=-]{wikinger-workbook-back.pdf}
\thispagestyle{empty}

\end{document}